\section{A Brief History of TEEs}
\glspl{tee} were first defined by the \gls{omtp} and ratified in 2009 \cite{Confidential2009} specifically for ``Handset manufacturers''. The \gls{omtp} standard was transferred to the Wholesale Applications Community (WAC) in 2010 and in July 2012 WAC itself was closed, with the \gls{omtp} standards being transferred to The GSM Association (originally Groupe Spécial Mobile)\cite{WAC}. In this paper we will discuss the two most prevalent implementations of this standard for the x86-64 and AArch64 architectures, as well as a completely open source hardware and software implementation of a TEE for the RISC-V architecture.

We're going to talk about \gls{ime}, \gls{sgx}, \gls{soc}, \gls{smm}, \gls{aes}, \gls{vtx}, \gls{spmp}, \gls{epmp}, \gls{iopmp}, and \gls{pmp}.


\begin{table}
\renewcommand\arraystretch{2}
\begin{tabular}{@{\,}r <{\hskip 2pt} !{\foo} >{\raggedright\arraybackslash}p{10cm}}
\toprule
\addlinespace[1.5ex]
1990 & Intel releases \gls{smm} \\
2005 & First processor using \gls{vtx} released \\
2006 & \textcolor{red}{Firmware vulnerabilities in \gls{smm} found} \\
2008 & \gls{ime} released \linebreak \textcolor{red}{First VM Escape CVEs found} \\
2009 & \textcolor{blue}{\gls{omtp} \glspl{tee} spec and \gls{tpm} ISO standard} \\
2011 & \textcolor{blue}{GlobalPlatform \gls{tee} spec} \\
2012 & \textcolor{blue}{ARM Cortex-A5 with TrustZone} \\
2013 & \textcolor{red}{TrustZone vulnerability found} \linebreak \textcolor{blue}{Intel introduces \glsreset{sgx}\gls{sgx}} \\
2015 & \textcolor{blue}{Intel Skylake with \gls{sgx}} \\
2016 & \textcolor{red}{\gls{ime} rootkit CVE published} \\
2017 & \textcolor{blue}{RISC-V \gls{pmp} spec ratified} \\
2018 & \textcolor{red}{\gls{sgx} attacks published} \\
2019 & \textcolor{blue}{\gls{soc} released with \gls{pmp}} \\
2020 & \textcolor{blue}{RISC-V \gls{epmp}, \gls{spmp}, \gls{iopmp} drafted} \\
\end{tabular}
\vspace{5mm}
\caption[Hardware Security Timeline]{\textbf{An overview of modern hardware security features, specifications, and vulnerabilities}In this timeline, events pertaining to TEEs are in \textcolor{blue}{blue} and vulnerabilities in hardware security technologies are in \textcolor{red}{red}. Dates of vulnerabilities are not exact, see \url{https://cve.mitre.org/} for exact dates and severity. Dates of technology releases are estimates and taken by the first broadly available product release with the given feature available.}
\label{tab:mod_sec_hist}
\end{table}


 We're going to cover how each of these technologies contributed to \gls{tee} development, how they effect their use, why they lead to the development of \glspl{tee}, and how they fall short of providing the security of a \gls{tee}. Moreover, we should cover the fact that attacks have already been seen in \glspl{tee}, and that mitigating those attacks is required. 

\section{From Handsets to IoT to Cloud}
