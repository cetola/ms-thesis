\section{The RISC-V Open Source ISA}
RISC-V is the fifth generation of RISC processors, which have been in development since the 1980s at the University of California at Berkeley. Hence the name RISC-V has the ``five'' spelled out as a Roman numeral. One cannot talk about RISC-V's history without mentioning that of the MIPS ISA, which began in the same timeframe at Stanford in the 1980s. While MIPS was very popular in its own right, RISC was the inspiration for many ISAs including Sun Microsystems’ SPARC line, DEC’s Alpha line, Intel’s i860 and i960 processors, and indeed the ARM processors that most of us now carry in our pockets.

RISC-V is by no means the first open source ISA. Sun Microsystems introduced the OpenSPARC project in 2005 using the GNU Public License (GPL)\cite{anemaet2003microprocessor}. As many students of SoC architecture classes will know, the MIPS architecture was provided under an ``open use'' license in a pilot program by Wave Computing allowing its use in educational settings without licensing fees. ARM has partly opened its architecture allowing specific partners to proposed changes to the ISA they license. Many of these changes in industry practices reflect a growing need for openness in hardware. As Asanovi{\'c} notes, ``While instruction set architectures (ISAs) may be proprietary for historical or business reasons, there is no good technical reason for the lack of free, open ISAs'' \cite{asanovic2014instruction}.

Care must be taken not to assume that the value found in open source software can be simply replicated in the hardware world. However, there is value to be gained for certain, and many of the same foundations underpinning open source software development as well as many of the lessons learned by those communities can be applied in hardware. Red Hat's Patent Promise and the growing problem that international regulations introduce are just two examples \cite{amye2021}. The fundamental concept of extensibility is perhaps the most obvious cross-cutting concern between open source hardware and software. Tim O'Reilly mentions the importance of extensibility to the proliferation of open source software 1990's. As O'Reilly notes, ``According to Linus Torvalds, Linux has succeeded at least in part because it followed good design principles, which allowed it to be extended in ways that he didn’t envision when he started work on the kernel. Similarly, Larry Wall explains how he created Perl in such a way that its feature-set could evolve naturally, as human languages evolve, in response to the needs of its users.'' \cite{o1999lessons}. Key to RISC-V's success will be the proliferation of open source extensions, tools, and designs that anyone can use as a platform on which to produce and launch innovative hardware.

RISC-V \gls{pmp} is developed as part of the ``TEE Task Group'' of RISC-V International. While membership in RISC-V International is required to participate in the development of specifications, membership is free for individuals and open to anyone. Specifications are developed using public mailing lists, and a review of each specification is open to a public review period before ratification. The specifications themselves are stored on a public GitHub repository where anyone can view the text during its development. RISC-V PMP was added to the Privileged ISA Specification \cite{PrivIsa2019} in 2019. Several libraries have been introduced to take advantage of this technology including: Multizone \cite{pinto2019industry}, Sanctum \cite{Costan2016a}, TIMBER-V \cite{weiser2019timber}, Mi6 \cite{bourgeat2019mi6}, and Keystone Enclave \cite{lee2019keystone, lee2020keystone, cheangverifying}. We will concentrate on Keystone Enclave as it is open source and still in active development.

\section{The RISC-V Memory Model}
\section{RISC-V Physical Memory Protection}
\section{Keystone Enclave}
\section{The future of RISC-V Memory Protection}