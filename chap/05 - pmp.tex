\section{The RISC-V Open Source ISA}
RISC-V is the fifth generation of RISC processors, which have been in development since the 1980s at the University of California at Berkeley. Hence the name RISC-V has the ``five'' spelled out as a Roman numeral. One cannot talk about RISC-V's history without mentioning that of the MIPS ISA, which began in the same timeframe at Stanford in the 1980s. While MIPS was very popular in its own right, RISC was the inspiration for many ISAs including Sun Microsystems’ SPARC line, DEC’s Alpha line, Intel’s i860 and i960 processors, and indeed the ARM processors that most of us now carry in our pockets.

RISC-V is by no means the first open source ISA. Sun Microsystems introduced the OpenSPARC project in 2005 using the GNU Public License (GPL)\cite{anemaet2003microprocessor}. As many students of SoC architecture classes will know, the MIPS architecture was provided under an ``open use'' license in a pilot program by Wave Computing allowing its use in educational settings without licensing fees. ARM has partly opened its architecture allowing specific partners to proposed changes to the ISA they license. Many of these changes in industry practices reflect a growing need for openness in hardware. As Asanovi{\'c} notes, ``While instruction set architectures (ISAs) may be proprietary for historical or business reasons, there is no good technical reason for the lack of free, open ISAs'' \cite{asanovic2014instruction}. RISC-V is also not an example of open source hardware as there is nothing inherent in the specifications encouraging the end product to be non-proprietary. However, it is certainly an enabler for open source hardware development by facilitating the sharing of ideas in the form of ISA extensions and hardware tooling.

Care must be taken not to assume that the value found in open source software can be simply replicated in the hardware world. However, there is value to be gained for certain, and many of the same foundations underpinning open source software development as well as many of the lessons learned by those communities can be applied in hardware. Red Hat's Patent Promise and the growing problem that international regulations introduce are just two examples \cite{amye2021}. The fundamental concept of extensibility is perhaps the most obvious cross-cutting concern between open source hardware and software. Tim O'Reilly mentions the importance of extensibility to the proliferation of open source software 1990's. As O'Reilly notes, ``According to Linus Torvalds, Linux has succeeded at least in part because it followed good design principles, which allowed it to be extended in ways that he didn’t envision when he started work on the kernel. Similarly, Larry Wall explains how he created Perl in such a way that its feature-set could evolve naturally, as human languages evolve, in response to the needs of its users.'' \cite{o1999lessons}. Key to RISC-V's success will be the proliferation of open source extensions, tools, and designs that anyone can use as a platform on which to produce and launch innovative hardware.

RISC-V \gls{pmp} is developed as part of the ``TEE Task Group'' of RISC-V International. While membership in RISC-V International is required to participate in the development of specifications, membership is free for individuals and open to anyone. Specifications are developed using public mailing lists, and a review of each specification is open to a public review period before ratification. The specifications themselves are stored on a public GitHub repository where anyone can view the text during its development. RISC-V PMP was added to the Privileged ISA Specification \cite{PrivIsa2019} in 2019. Several libraries have been introduced to take advantage of this technology including: Multizone \cite{pinto2019industry}, Sanctum \cite{Costan2016a}, TIMBER-V \cite{weiser2019timber}, Mi6 \cite{bourgeat2019mi6}, and Keystone Enclave \cite{lee2019keystone, lee2020keystone, cheangverifying}. We will concentrate on Keystone Enclave as it is open source and still in active development.

\section{The RISC-V Memory Model}
RISC-V hardware threads, which are referred to in the specification as ``harts'' \cite{UnprivIsa2019}, have a byte-addressable address space of 2\textsuperscript{XLEN} bytes for all memory accesses. Here, XLEN refers to the width of a register in either 32 or 64 bits. Words are defined as having 32 bits. Halfwords are 16 bits, doublewords are 64 bits (8 bytes), and so on. The address space can be thought of as a ring, so that the last memory address is adjacent to the first. Memory instructions will simply wrap around the space ignoring that they have effectively ``walked off the end''. The specification leaves room for virtual memory by allowing for both explicit and implicit stores and loads. Chapter 3 defines the ``Zifencei'' extension, which defines a fence instruction that explicitly synchronizes writes to instruction memory and instruction fetches on the same \gls{hart}. The order in which these loads and stores happen is up to the implementation, and the rules for consistency are defined in the memory model.

The memory model of an instruction set architecture defines the values returned by a load. The main memory model for RISC-V is the \gls{rvwmo} model, ``which is designed to provide flexibility for architects to build high-performance scalable designs while simultaneously supporting a tractable programming model'' \cite{UnprivIsa2019}. There is also the ``Ztso'' extension which gives us the RISC-V Total Store Ordering (RVTSO) memory consistency model. One can think of RVTSO as the ``stronger'' model when compared to \gls{rvwmo}. We will only cover \gls{rvwmo} briefly here as knowledge of RVTSO is not required to understand \gls{pmp}, nor is a deep understanding of the \gls{rvwmo} required.

The microarchitecture implementing any given memory model is only required by the architecture to follow the memory model rules set forth in the model. This model does not set any other regulation on how the implementation achieves those rules, be the implementation speculative or not, in order or out of order, multithreaded or not, etc. As such, the \gls{rvwmo} starts by defining a set of primitives it uses to define the model. The base primitives are load and store, defined in Chapter 8 which covers the ``A'' Standard Extension for Atomic Instructions \cite{UnprivIsa2019}. It then uses a combination of \gls{axiomatic} and \gls{operational} semantics to define how memory consistency should be achieved.

The memory model defines the syntactic dependencies of memory operations. This is essentially a way to understand what the differences are between memory operations and the instructions which generate those operations. The memory model then defines thirteen rules that allow for the program order of each \gls{hart} to be consistent with the global order of all operations, called the ``preserved program order''. The rules cover overlapping address ordering, explicit synchronization, syntactic dependencies, and pipeline dependencies. The model then defines three axioms: the Load Value Axiom, the Atomicity Axiom, and the Progress Axiom. Any implementation which follows the \gls{rvwmo} memory model must conform to the thirteen rules and satisfy the three axioms.

\section{RISC-V Physical Memory Attributes}
RISC-V system's physical memory map has various properties and capabilities that are described in detail in the Privileged Specification \cite{PrivIsa2019} and referred to as \gls{pma}. These attributes determine things like read, write, and execute permissions of a specific region of physical memory. Most systems build on the RISC-V architecture will require that the \glspl{pma} are checked later in the pipeline and that this check is done in hardware. This is in contrast to other architectures where these types of checks are done in virtual page tables where the \gls{tlb} contains the information the pipeline needs regarding these attributes. RISC-V systems call the mechanism for making these checks the ``\gls{pma} checker'' and many physical attributes will be hardwired into the checker when designing the chip. For those attributes that are not known at design time there are special platform specific control registers that can be configured at runtime.

Memory regions are given attributes based on if they are part of main memory or part of I/O. The access width of a region can be anything from 8-bit byte to long multi-word bursts. As the specification states, ``Complex atomic memory operations on a single memory word or doubleword are performed with the
load-reserved (LR) and store-conditional (SC) instructions.'' \cite{PrivIsa2019}. RISC-V also allows for special \glspl{amo} which are instructions that perform operations for multiprocessor synchronization. A given \gls{pma} will contain information regarding which atomic operations are allowed for a specific region of memory.  Table \ref{table:riscv-pma-amo} lists the \gls{amo} available to I/O as of this writing.

\renewcommand{\arraystretch}{1.5}
\begin{table*}[htp]
\begin{center}
\begin{tabular}{|l|l|}
  \hline
  AMO Class & Supported Operations \\
  \hline
  AMONone       & {\em None} \\
  AMOSwap       & {\tt amoswap} \\
  AMOLogical    & above + {\tt amoand}, {\tt amoor}, {\tt amoxor} \\
  AMOArithmetic & above + {\tt amoadd}, {\tt amomin}, {\tt amomax}, {\tt amominu}, {\tt amomaxu} \\
  \hline
\end{tabular}
\end{center}
\caption[RISC-V Atomic Instructions for I/O]{\textbf{Classes of AMOs supported by I/O regions.} Reproduced from the original RISC-V Privileged Specification \cite{PrivIsa2019}.}
\label{table:riscv-pma-amo}
\end{table*}

As mentioned earlier, regions of memory will be classified either as main memory or as I/O, and this must be considered by the FENCE instruction when ordering memory. Regardless of if the memory model used is \gls{rvwmo} or RVTSO, memory regions may be classified as either having relaxed or strong ordering. Strongly ordered memory regions use a ``channel'' mechanism to guarantee ordering. Using \glspl{pma}, systems can decide to set the type of ordering dynamically or not. The specification requires that all regions of memory be coherent, such that any change made by one agent to a memory region must eventually be visible other agents of the system. Cacheability is left up to the platforms, however three types are called out: ``master-private, shared, and slave-private'' \cite{PrivIsa2019}. \glspl{pma} will describe the specific cache features of each region as well as if the region is idempotent.

\section{RISC-V Physical Memory Protection}
Unlike \glspl{pma}, RISC-V \glsreset{pmp}\gls{pmp} consists of a set of configurations that can be changed dynamically during runtime. The privileged specification describes a ``\gls{pmp} unit'' as a set of ``per-hart machine-mode control registers to allow physical memory access privileges (read, write, execute) to be specified for each physical memory region.'' \cite{PrivIsa2019}. These registers must be checked in parallel with the \gls{pma} attributes described int he previous section.

\renewcommand{\arraystretch}{2}
\begin{table}[t]
\centering
\begin{tabular}{|l|l|l|}
\hline
\textbf{Processor Mode} & \textbf{Description} \\ \hline
U Mode & User Processor Mode \\ \hline
S Mode & Supervisor Processor Mode \\ \hline
H Mode & Hypervisor Processor Mode (Draft) \\ \hline
M Mode & Machine Processor Mode \\ \hline
\end{tabular}
\caption[RISC-V Processor Modes]{\textbf{RISC-V Privilege Level Mapping} is similar to Arm's exception level mapping in that we have four modes. The most privileged machine mode is the only required mode.}
\label{table:rv-priv}
\end{table}


As we discussed in the last two chapters, processor privilege levels or ``modes'' are a critical foundation to our security model. RISC-V \gls{pmp} allows for specific registers that are only available to the highest privileged machine-mode or m-mode. The different processor modes for RISC-V are described briefly in Table \ref{table:rv-priv}. Note that H Mode is still only in the draft state, and is not applicable when discussing \gls{pmp}. While \gls{pmp} is available in both 64 bit and 32 bit versions, we will only describe 32 bit configurations in this thesis. It is enough for our comparison that we understand that a 64 bit implementation is possible and holds to the same or better standards as the 32 bit version. 

\renewcommand{\arraystretch}{1}
\vspace{.1cm}
\begin{figure}[hpt]
{\footnotesize
\adjustbox{minipage=1.3em,valign=t}{\subcaption{}\label{sfig:pmpcfg}}
\begin{subfigure}[b]{\textwidth}
\begin{center}
\begin{tabular}{@{}Y@{}Y@{}Y@{}Yl}
\instbitrange{31}{24} &
\instbitrange{23}{16} &
\instbitrange{15}{8} &
\instbitrange{7}{0} & \\
\cline{1-4}
\multicolumn{1}{|c|}{pmp3cfg} &
\multicolumn{1}{c|}{pmp2cfg} &
\multicolumn{1}{c|}{pmp1cfg} &
\multicolumn{1}{c|}{pmp0cfg} &
\tt pmpcfg0 \\
\cline{1-4}
8 & 8 & 8 & 8 & \\
\instbitrange{31}{24} &
\instbitrange{23}{16} &
\instbitrange{15}{8} &
\instbitrange{7}{0} & \\
\cline{1-4}
\multicolumn{1}{|c|}{pmp7cfg} &
\multicolumn{1}{c|}{pmp6cfg} &
\multicolumn{1}{c|}{pmp5cfg} &
\multicolumn{1}{c|}{pmp4cfg} &
\tt pmpcfg1 \\
\cline{1-4}
8 & 8 & 8 & 8 & \\
\instbitrange{31}{24} &
\instbitrange{23}{16} &
\instbitrange{15}{8} &
\instbitrange{7}{0} & \\
\cline{1-4}
\multicolumn{1}{|c|}{pmp11cfg} &
\multicolumn{1}{c|}{pmp10cfg} &
\multicolumn{1}{c|}{pmp9cfg} &
\multicolumn{1}{c|}{pmp8cfg} &
\tt pmpcfg2 \\
\cline{1-4}
8 & 8 & 8 & 8 & \\
\instbitrange{31}{24} &
\instbitrange{23}{16} &
\instbitrange{15}{8} &
\instbitrange{7}{0} & \\
\cline{1-4}
\multicolumn{1}{|c|}{pmp15cfg} &
\multicolumn{1}{c|}{pmp14cfg} &
\multicolumn{1}{c|}{pmp13cfg} &
\multicolumn{1}{c|}{pmp12cfg} &
\tt pmpcfg3 \\
\cline{1-4}
8 & 8 & 8 & 8 & \\
\end{tabular}
\end{center}
\end{subfigure}
}
{\footnotesize
\adjustbox{minipage=1.3em,valign=t}{\subcaption{}\label{sfig:pmpaddr}}
\begin{subfigure}[b]{\textwidth}
\begin{center}
\begin{tabular}{@{}J}
\instbitrange{31}{0} \\
\hline
\multicolumn{1}{|c|}{address[33:2] (\warl)} \\
\hline
32 \\
\end{tabular}
\end{center}
\end{subfigure}
}
{\footnotesize
\adjustbox{minipage=1.3em,valign=t}{\subcaption{}\label{sfig:pmpcfglayout}}
\begin{subfigure}[b]{\textwidth}
\begin{center}
\resizebox{.85\textwidth}{!}{
\begin{tabular}{YSSYYY}
\instbit{7} &
\instbitrange{6}{5} &
\instbitrange{4}{3} &
\instbit{2} &
\instbit{1} &
\instbit{0} \\
\hline
\multicolumn{1}{|c|}{L (\warl)} &
\multicolumn{1}{c|}{\wiri} &
\multicolumn{1}{c|}{A (\warl)} &
\multicolumn{1}{c|}{X (\warl)} &
\multicolumn{1}{c|}{W (\warl)} &
\multicolumn{1}{c|}{R (\warl)}
\\
\hline
1 & 2 & 2 & 1 & 1 & 1 \\
\end{tabular}
}
\end{center}
\end{subfigure}
}
\caption[RISC-V 32 bit PMP CSR Layout and Format]{\textbf{RV32 PMP CSR layout and format as well as the address register format.} Figure \ref{sfig:pmpcfg} describes the RV32 PMP configuration CSR layout. Figure \ref{sfig:pmpaddr} is the PMP address register format for RV32. Lastly, Figure \ref{sfig:pmpcfglayout} shows the PMP configuration register format. Figures reproduced from the RISC-V privileged specification \cite{PrivIsa2019}.}
\label{fig:pmpcfg-rv32}
\end{figure}

There are currently up to 16 \gls{pmp} configuration registers available in both 32 and 64 bit modes and their layout for 32 bit is shown in Figure \ref{fig:pmpcfg-rv32}. The figure also describes the layout of those configuration registers where\;\warl\;is Write-Any Read-Legal and\;\wiri\;is Write-Ignored Read-Ignored. This is to say that the two bits 5 and 6 are simply ignored in this register as they are reserved. All the other bits will follow the RISC-V common\;\warl\;as defined in the CSR section of the specification. There is also an address register format defined in Figure \ref{fig:pmpcfg-rv32}, which is the starting address of the \gls{pmp} region. No ``stop address'' is required, and if only one \gls{pmp} region is defined, it will encompass the entire memory space. Indeed when many systems boot they may choose to create a single \gls{pmp} region with read/write/execute enabled for all modes as a default setting.

The ``L'' bit of the configuration register defines if that region of memory is locked and cannot be read, written, or executed from, regardless of the processor mode. A reset is required for m-mode to once again control that CSR and allow that region's permissions to change. As mentioned, bits 6 and 5 are reserved. Bits 4 and 3 define the type of address matching scheme that will be used to match the store or load address range against the \gls{pmp} address register. The options for this two bit value are OFF, Top of Range (TOR), Naturally Aligned 4-byte region (NA4), or Naturally Aligned Power of Two (NAPOT). If set to NAPOT the granularity can be set by the system to any NAPOT value greater than or equal to 8. The last three bits are set to 1 or 0 for read, write, and execute, with 1 meaning ``enabled'' and 0 meaning ``disabled''. These bits currently apply to both S Mode and U Mode, though we will discuss the future of S Mode \gls{pmp} in a following section.

Using these configuration registers and address registers, RISC-V systems are able to create regions of memory with simple read, write, and execute privileges enforced by hardware. This allows systems to build secured areas of memory similar to that of both Intel \gls{sgx} and Arm TrustZone. However, without a framework on which to build these secured areas of memory, the specification only enables hardware engineers to create a foundation on which to build a \gls{tee}. We will now examine Keystone Enclave, an open source framework for building hardware \glspl{tee} using RISC-V \gls{pmp} as the foundation.

\section{Keystone Enclave}
\section{Future Extensions of RISC-V Memory Protection}