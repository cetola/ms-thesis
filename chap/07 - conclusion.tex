This thesis has described a method that allows for a comprehensive and rigorous comparison of \glsreset{tee}\glspl{tee}. This method involves identifying the properties of interest for hardware architects, firmware authors, and \gls{tee} software designers alike. Key to our comparison of the implementations of those properties is breaking each property down into relevant constituent parts. The relevancy of each property will depend on the use case of the designer, and should be taken into account during comparison. This method does not provide us with which technology is the ``best,'' but rather gives us insight into which \gls{tee} might best fit a specific set of needs.

This proposed method will save hardware architects the frustration that comes with a wide range of technology choices and it will save countless engineer-hours wasted if an inadequate technology is chosen. It is possible that this method may not be appropriate for some future version of a \gls{tee}, as evidenced by how new technologies like RISC-V \glsreset{pmp}\gls{pmp} are both mimicking older technologies as well as developing new and innovative features. However, as the use of \glspl{tee} becomes more common and these technologies take on new unique properties, it is likely that these new properties will be easily integrated into this method of comparison.