This thesis has described a method that allows for a comprehensive and rigorous comparison of \glsreset{tee}\glspl{tee}. This method involves identifying the properties of interest for hardware architects, firmware authors, and \gls{tee} software designers alike. Key to our comparison of the implementations of those properties is breaking each property down into relevant constituent parts. The relevancy of each property will depend on the use case of the designer, and should be taken into account during comparison. This method does not provide us with which technology is the ``best,'' but rather gives us insight into which \gls{tee} might best fit a specific set of needs.

There are several use cases for systems requiring a \gls{tee} that this thesis does not take into account. These use cases could be the subject of future work in improving this methodology. Multi-socket processors \cite{knauth2018integrating} add another layer of complexity to \glspl{tee} and should be considered for cloud compute or data center use cases. Likewise systems might require multiple \glspl{tee} or heterogeneous \gls{tee} environments, perhaps for enhanced security use cases. As an example, a system running both TrustZone alongside RISC-V \gls{pmp} may require communication between these different \glspl{tee} to assure secure computation. Examining the details of how these systems are able to interact and communicate allows for new avenues of comparison. Future work covering these use cases would add to the rigorous nature of this method. 

This proposed method will save hardware architects the frustration that comes with a wide range of technology choices. The method provides an organizational framework for characterizing properties of a \gls{tee} and provides value to product designers who must focus on possible use cases ranging from mobile devices to cloud services. It is possible that this method may be insufficient for some future version of a \gls{tee}, as evidenced by how new technologies like RISC-V \glsreset{pmp}\gls{pmp} are both mimicking older technologies as well as developing new and innovative features. However, as the use of \glspl{tee} becomes more common and these technologies take on new unique properties, it is likely that these new properties will be easily integrated into this method of comparison.