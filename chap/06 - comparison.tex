\section{A Method for Comparing TEEs}
As mentioned in Chapter \ref{chap:intro}, we will use the properties of a \gls{tee} as defined by the Confidential Computing Consortium \cite{cccTAC}, an open source project that brings together hardware vendors, cloud providers, and software developers to accelerate the adoption of \gls{tee} technologies and standards. They define code integrity, data integrity, and data confidentiality as three required properties of any hardware \gls{tee}. They also discuss more advanced features like code confidentiality, authenticated launch, programmability, \gls{attestation}, and recoverability, which can be considered optional properties of a \gls{tee}. Notably absent from our list of required properties is code confidentiality. There are several reasons why we might want our code to be confidential. It may contain proprietary algorithms for machine learning or other intellectual property. However the code that runs inside the \gls{tee} may be open source, making the feature of code confidentiality counterproductive. As such, that feature is considered an optional property by our definition.

By authenticated launch, we mean a mechanism by which the \gls{tee} can prevent a host application from loading, or limit its functionality based on a set of criteria or security model. By programmability, we mean the ability of a \gls{tee} to be programmed with code that is unrestricted and able to be changed by the end user. This is apposed to a \gls{tee} with a limited set of available functions or one that is pre-programmed by the manufacturer or system vendor. By recoverability, we mean a way for the \gls{tee} to ``roll back'' to some known good state should any of the validations performed in order to assure integrity or confidentiality fail.

While these properties are a good starting point for analysis and comparison, we must break these down further in order to engage in a more rigorous comparison. Authenticated launch can be broken up into two parts: the ability to prevent application launch, or the ability to launch applications with limited functionality. Programmability will be broken into three types: fully programmable, partially programmable, or non-programmable. \Gls{attestation} can be divided into three parts, with remote attestation, local attestation, and if a hardware root of trust is available. Recoverability will be considered. Revocation of keys or other components should cause those associate signatures to no longer be trusted. If we are able to recover the \gls{tcb} using a hardware \gls{rot} then the system is considered to be recoverable. This involves re-issuing keys from some trusted source without the need for re-provisioning.

\renewcommand{\arraystretch}{1}
\begin{table*}[p]
\begin{center}
\begin{tabular}{|l|l|l|l|}
\hline
\textbf{}                                  & \cellcolor{tbl-gre}\textbf{Intel SGX} & \cellcolor{tbl-gre}\textbf{Arm TrustZone} & \cellcolor{tbl-gre}\textbf{RISC-V PMP}   \\ \hline
\cellcolor{tbl-yel}\textbf{Attestation}          & Platform dependent (Intel, Supermicro)                        & SoC manufacturer dependent (Qualcomm)                   & Firmware (Keystone, Sanctum)             \\ \hline
\cellcolor{tbl-yel}\textbf{Authenticated Launch} & Platform dependent (Intel)                        & Firmware dependent (TF-A)                            & Firmware dependent (none, currently)          \\ \hline
\cellcolor{tbl-yel}\textbf{Code Confidentiality} & Provided by SGX                            & Platform Dependent  (MCIMX8M-EVKB)                           & Provided by PMP                               \\ \hline
\cellcolor{tbl-yel}\textbf{Programmable}         & Provided by SGX                            & Provided by TrustZone                          & Firmware (Keystone)                      \\ \hline
\cellcolor{tbl-yel}\textbf{Recoverable}          & Platform dependent (Supermicro)                      & Firmware dependent  (TF-A)                          & Firmware dependent (none, currently) \\ \hline
\end{tabular}
\end{center}
\caption[TEE Features and their dependencies]{\textbf{Several optional TEE features and the dependencies of those features.} Optional features of a TEE can be provided by the technology itself, by the chip manufacturer, by the platform vendor, or by the firmware. In turn, the firmware can be provided to the end user or can be custom. TEE Technologies are colored in \colorbox{tbl-gre}{green} while properties of the TEE are colored in \colorbox{tbl-yel}{yellow}.}
\label{tab:tee-compare}
\end{table*}

\begin{table*}[p]
\begin{center}
\begin{tabular}{|l|l|l|l|}
\hline
\textbf{}                                           & \cellcolor{tbl-gre}\textbf{Intel SGX} & \cellcolor{tbl-gre}\textbf{Arm TrustZone} & \cellcolor{tbl-gre}\textbf{RISC-V PMP} \\ \hline
\cellcolor{tbl-yel}\textbf{Remote Attestation} & Intel dependent                            & SoC manufacturer dependent (Samsung)                     & Not currently available                     \\ \hline
\cellcolor{tbl-yel}\textbf{Local Attestation}  & Inherent in SGX                            & Firmware dependent (TF-A)                            & Firmware dependent (Keystone)                          \\ \hline
\cellcolor{tbl-yel}\textbf{Hardware RoT}       & Platform dependent (Intel, Supermicro)                       & Platform dependent  (MCIMX8M-EVKB)                           & Inherent in PMP specification                   \\ \hline
\end{tabular}
\end{center}
\caption[Attestation Comparison]{\textbf{The possible attestation features and the dependencies of those features.} Optional features of attestation can be provided by the technology itself. However, several features require either the chip manufacturer, the platform vendor, or the end user to provide some resources. TEE Technologies are colored in \colorbox{tbl-gre}{green} while properties of the TEE are colored in \colorbox{tbl-yel}{yellow}.}
\label{tab:attest-compare}
\end{table*}

\begin{table*}[p]
\begin{center}
\begin{tabular}{|l|l|l|l|}
\hline
\textbf{}                                                         & \cellcolor{tbl-gre}\textbf{Intel SGX} & \cellcolor{tbl-gre}\textbf{Arm TrustZone} & \cellcolor{tbl-gre}\textbf{RISC-V PMP} \\ \hline
\cellcolor{tbl-yel}\textbf{Full recovery, any state}         & Y                                          & N                                              & N                                           \\ \hline
\cellcolor{tbl-yel}\textbf{Partial recovery, any state}      & Y                                          & Y                                              & N                                           \\ \hline
\cellcolor{tbl-yel}\textbf{Full recovery, limited states}    & N                                          & Y                                              & N                                           \\ \hline
\cellcolor{tbl-yel}\textbf{Partial recovery, limited states} & N                                          & Y                                              & Y                                           \\ \hline
\end{tabular}
\end{center}
\caption[Recoverability Comparison]{\textbf{The possible recoverability features and the dependencies of those features.} Note that all these features will be firmware dependent. For Intel SGX, the features are supported as part of the Intel SDK provided to customers. Arm TrustZone has a reference implementation of the firmware required in Arm Trusted Firmware (see: \url{https://www.trustedfirmware.org/projects/tf-a/}). RISC-V only has partial recoverability and can be found in the Keystone Project here: \url{https://keystone-enclave.org/}. TEE Technologies are colored in \colorbox{tbl-gre}{green} while properties of the TEE are colored in \colorbox{tbl-yel}{yellow}.}
\label{tab:recover-compare}
\end{table*}



\section{TEEs in Cloud Computing}
\section{Embedded and IoT}
\section{Considerations for Othe Use Cases}
