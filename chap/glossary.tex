 \newglossaryentry{attestation}
{
        name=attestation,
        description={The process of assuring the validity and security of a system. The system can only be valid and secure if the code and data stored on the system have not been altered in any way that either causes the system to operate outside specifications or compromises the trust model}
}

\newglossaryentry{measurement}
{
        name=cryptographic hash,
        description={A cryptographic hash is the result of a mathematical function whose input is data of variable length and whose output is data of a deterministic value and fixed length. If this hash represents a piece of software, we can consider this hash the software's ``identity'' \cite{beekman2016improving} for purposes of attestation. Many systems of attestation refer to this hash as a ``measurement''}
}

\newglossaryentry{root of trust}{
    name={Root of Trust},
    description={``A computing engine, code, and possibly data, all co-located on the same platform which provides security services. No ancestor entity is able to provide trustworthy attestation for the initial code and data state of the Root of Trust'' \cite{GlobalPlatform2018}. For system security purposes one might say, ``the buck stops here''}
}

\newglossaryentry{chain of trust}{
    name={chain of trust},
    description={A series of entities that engage in secure transactions in order to provide a service. The first entity in the chain is referred to as the root of trust, while the last entity in the chain is often the end user or application requiring a secure transaction}
}

\newglossaryentry{axiomatic}{
    name={axiomatic},
    description={A semantic of logic which formalizes the definition of a concept using a set of criteria or axioms, all of which must be true for the given definition to be met}
}

\newglossaryentry{operational}{
    name={operational},
    description={A semantic of logic which formalizes the definition of a concept by generating a golden output model. Any system meeting the definition must produce the same output as that defined in the model}
}

\newglossaryentry{hart}{
    name={hart},
    description={In a RISC-V system, a core contains an independent instruction fetch unit and each core can have multiple hardware threads. These hardware threads are referred to as harts}
}

\newglossaryentry{ring}{
    name={privilege ring},
    plural={privilege rings},
    description={Also known as a ``protection ring'' or ``protection domain'' \cite{6234805}, these modes of operation allow a processor to restrict access to memory or special instructions. Switching between rings is the function of low-level software or firmware. Before the problem of secure remote computation described in Chapter \ref{chap:intro}, these rings provided adequate protection for software applications}
}

\newglossaryentry{trusted compute base}{
    name={Trusted Compute Base},
    description={When referenced generally, the code which must be trusted in order for the system to be considered secure. The code can include platform firmware, firmware from the manufacturer, or any code running inside the \gls{tee}. When in reference to a specific application, the TCB may only refer to the part of the application which runs inside the TCB}
}
