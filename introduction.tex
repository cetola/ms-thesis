\section{Problem Statement}
Historically, computer architecture security relied on processor modes or privilege modes where code was allowed to execute. In these modes separation of privileges is achieved and often referred to as “rings” with “ring 0” being the most privileged (machine mode, kernel code) and ring 3 being the least privileged (user mode, application code). Device drivers run in the rings between these two modes with virtualization being the most common use case\footnote{The virtual machine’s kernel, for example, runs in ring 1, one level up from machine mode.}. As applications became more complex, specifically with the advent of large-scale virtualization and the internet, this simple security model broke down as executed code could no longer be trusted, nor its origin verified. The problem of “secure remote computation” arises where the data owner must trust not only the software provider, but also the remote computer and infrastructure on which that software is executed. Homomorphic encryption solves this problem to some extent, however the performance overhead of this transaction limits its application\cite{Gentry01}. 
In an attempt to address these issues, microprocessor designers have implemented versions of a “Trusted Execution Environment”, first defined by the Open Mobile Terminal Platform and ratified in 2009\cite{OMTP01}. The OMTP standard was transferred to the Wholesale Applications Community (WAC) in 2010 and in July 2012 WAC itself was closed, with the OMTP standards being transferred to The GSM Association (originally Groupe Spécial Mobile)\cite{WAC}. In this paper we will discuss the two most prevalent implementations of this standard for the x86-64 and AArch64 architectures, as well as a completely open source hardware and software implementation of a TEE for the RISC-V architecture.
\section{The Intel SGX Solution}
Intel \gls{sgx} is built on designs of software attestation already proven in technologies like the \gls{tpm} and Intel \gls{txt}. In \gls{sgx}, these concepts of software attestation are used to create containerized sections of memory on the remote computer called “secure enclaves” where data and code can be loaded or executed securely. These enclaves are verified by both a cryptographic attestation key of the container’s contents as well as a hardware \gls{rot} manufacturer’s key. Unlike the \gls{tpm} and gls{txt} technologies, \gls{sgx} securely operates only on a small amount of data and code called the \gls{tcb}, leaving the majority of memory outside this \gls{tcb}.
\section{Initial SGX Enclave Setup}
\gls{sgx} sets aside a memory region, called the \gls{prm}. The CPU protects the \gls{prm} from all non-enclave memory accesses, including kernel, hypervisor and \gls{smm} accesses, and \gls{dma} from peripherals\cite{Costan2016IntelSE}. This memory is used to store the \gls{epc}, which are the 4kb pages holding both the enclave data and code. While untrusted system software both assigns these EPCs to an enclave and loads them with data, it is the CPU which keeps track of all the \gls{epc}s ensuring that they only belong to one enclave. Once the system software loads data into the enclave it asks the CPU to mark that enclave as initialized, after which no other data may be loaded into the enclave as this setup process is disabled for that enclave. After initialization, this enclave is measured by a cryptographic hash to ensure that any operations performed on the enclave are done so in a secure environment.
\section{Executing SGX Enclave Code}
Execution flow can only move into an enclave via a special CPU instruction, much like switching from user mode to kernel mode. The actual execution happens in user mode and takes advantage of address translation from the Operating System or hypervisor. The CPU executing the enclave code performs an \gls{aex} whenever execution moves outside the enclave such as servicing an interrupt or during a page fault. The CPU state is saved inside the enclave before exiting ensuring that the CPU can security restore the state of execution. There are special machine mode CPU instructions that are used both in allocating \gls{epc} pages to the enclave as well as evicting those pages into untrusted DRAM. This facilitates code outside the enclave to operate on code within the enclave. \gls{sgx} uses cryptographic protections to assure the confidentiality, integrity and freshness of the evicted \gls{epc} pages while they are stored in untrusted memory\cite{Costan2016IntelSE}. In this way, Intel \gls{sgx} is able to allow a specific amount of code and data to remain protected while still allowing access to that data by code outside the trust boundary.
\section{The Arm TrustZone Solution}
When evaluating how Arm’s TrustZone works, we must remember several important distinctions. Firstly, the Arm specifications include several different architectures with several different states. Each Arm architecture and state combination may operate slightly differently in regard to how TrustZone is implemented. This paper will only consider the ARMv8-A architecture running in the AArch64 state. Secondly, hardware manufacturers may choose to implement security in many ways, and with much more flexibility than in Intel platforms. For simplicity’s sake, this paper will only cover standard Arm solutions for TrustZone implementation.
 
Arm \gls{soc} processors create a more absolute separation between the worlds of “secure” and “normal or insecure” operation than Intel \gls{sgx}. This is accomplished using three principal technologies on the bus, the \gls{soc} core, and the debug infrastructure. Firstly, the bus interface, called the \gls{amba} \gls{axi}, partitions all of the \gls{soc}’s hardware and software resources by taking advantage of a set of bits. Hardware logic present in this “TrustZone-enabled AMBA3 AXI” bus fabric ensures that no “secure world” resources can be accessed by “normal world” components. These bits include AWPROT for write transactions and ARPROT for read transactions (low is Secure and high is Non-secure). Secondly, \gls{soc}s using cores like the ARMv8-A include implemented extensions which enable a single physical processor core to safely and efficiently execute code from both the Normal world and the Secure world in a time-sliced fashion. Lastly, the security-aware debug infrastructure controls debug access to the secure world\cite{ArmWhitepaper}.

